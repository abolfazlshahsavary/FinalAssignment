\documentclass[12pt]{article}

\usepackage[utf8]{inputenc}
\usepackage{geometry}
\geometry{a4paper, margin=1in}
\usepackage{graphicx}
\usepackage{hyperref}
\usepackage{fancyhdr}

\setlength{\headheight}{15pt}
\pagestyle{fancy}
\fancyhf{}
\rhead{Computer Workshop Course}
\lhead{Final Assignment}
\rfoot{Page \thepage}

\title{Final project}
\author{abolfazl shahsavari}
\date{January 2024}

\begin{document}

\maketitle
\newpage
\tableofcontents
\newpage

\section{repository link}
\paragraph{\url{https://github.com/abolfazlshahsavary/FinalProject}}
\section{Git and GitHub}
\subsection{Repository Initialization and Commits}
\paragraph{Write about how you set up the repository for this assignment. Explain every step in detail?}
\paragraph{To set up a repository, first of all,we must have a GitHub account. After creating a GitHub account or logging into the account, we perform the following steps:}

\begin{itemize}
    \item Go to the repository section in your account
    \item Click on New Repository
    \item Select the repository owner and its name
    \item Enter the explation in the description field
    \item Choose whether your repository is public or private
    \item If you want to add a README to your repository and edit it later
    \item You can add license to tell others the ablity of your code and what they can do with your code
    \item Finally click on create repository 
\end{itemize}
\subsection{GitHub Actions for LaTeX Compilation}
\paragraph{Provide a walkthrough of setting up GitHub Actions to automatically compile your LaTeX
document and any challenges you encountered?}
\paragraph{To launch a compiler for LaTech with Github action, you must set up a workflow and enter your codes for compilation in the main.yml file.\\ Note that you must use tags in this work.
One of the challenges that I faced in this way was the creation of tags, which must first be created from your system and then pushed to the repository.}





\end{document}